\documentclass[11pt]{article}

\usepackage[pdftex]{graphicx}
\usepackage[cmex10]{amsmath}
\interdisplaylinepenalty=2500
\usepackage{amssymb}
\usepackage{url}
%\usepackage{mathptmx}		% for times fonts
\usepackage{color,xcolor,colortbl}
\usepackage{algorithm,algorithmic}
\usepackage[mathscr]{euscript}		% DO NOT remove this line, otherwise some symbols may get overload!
\usepackage[letterpaper]{geometry}
\geometry{verbose,tmargin=1in,bmargin=1in,lmargin=1in,rmargin=1in}

\usepackage{mdwmath}
\usepackage{mdwtab}
\usepackage[font=footnotesize]{subfig}

\definecolor{darkblue}{rgb}{0.0,0.0,0.5}
\usepackage[unicode=true,bookmarks=true,bookmarksnumbered=false,bookmarksopen=true,hidelinks=true,colorlinks=true,citecolor=darkblue,linkcolor=black]{hyperref}
	
\usepackage{epstopdf}
\usepackage{amsthm}
\usepackage{relsize}

\newcommand{\etal}{\textit{et~al.}}
\newcommand{\eg}{\textit{e.g.}}
\newcommand{\ie}{\textit{i.e.}}
\newcommand{\alg}[1]{\mbox{\textsf{#1}}}
\newcommand{\view}[1]{\textsf{{\small VIEW}}_{#1}}

\setlength{\parskip}{0.25em}
\let \labelindent \relax
\usepackage{enumitem}
\setlist[description]{listparindent=\parindent,leftmargin=0em,itemsep=1em}

\theoremstyle{mytheoremstyle}
\newtheorem{theorem}{Theorem}
\newtheorem{lemma}{Lemma}
\newtheorem{corollary}{Corollary}
\newtheorem{definition}{Definition}
\floatname{algoritheorem}{Protocol}

\begin{document}
	
\global\long\def\E{\mathrm{\mathbf{E}}}

\global\long\def\Var{\mathrm{\mathbf{Var}}}

\title{Notes on Randomized Algorithms}

\author{Mahdi Zamani\\ Visa Research, Palo Alto, CA \\ \textit{mzamani@visa.com}}
\date{}

\maketitle

\section*{Basics}
\begin{description}
	\item [{Poisson~Trials.}] A sequence of \emph{independent} 0-1 random
	variables $(X_{1},...,X_{n})$, where for every $i\in[n]$,
	\begin{eqnarray*}
		\Pr(X_{i}\text{ succeeds}) & = & \Pr(X_{i}=1)=p_{i}\\
		\Pr(X_{i}\text{ fails}) & = & \Pr(X_{i}=0)=1-p_{i}
	\end{eqnarray*}
	\[
	0\leq p_{i}\leq1.
	\]
	\item [{Bernoulli~Trials.}] A special case of Poisson trials, where the
	independent random variables have the same distribution, i.e., for
	every $i\in[n]$,
	\begin{eqnarray*}
		\Pr(X_{i}=1) & = & p\\
		\Pr(X_{i}=0) & = & 1-p
	\end{eqnarray*}
	\[
	0\leq p\leq1.
	\]
	\item [{Binomial~Distribution.}] The distribution of the number of successes
	in a sequence of Bernoulli trials, e.g., the number of heads in a
	sequence of $n$ coin flips. A binomial random variable $X$ with
	parameters $n$ and $p$, denoted by $B(n,p)$, is defined by the
	following probability distribution on $j=0,1,...,n$:
	\[
	\Pr(X=j)=\binom{n}{j}p^{j}(1-p)^{n-j}
	\]
	\[
	\E[X]=np.
	\]
	\item [{Geometric~Distribution.}] The distribution of the number of trials
	in a Bernoulli process until we get a success, e.g., the number of
	times to flip a coin until it lands on heads. A geometric random variable
	$X$ with parameter $p$ is defined by the following probability distribution:
	\[
	\Pr(X=n)=p(1-p)^{n-1}
	\]
	\[
	\E[X]=1/p.
	\]
	\item [{Poisson~Distribution.}] A Poisson random variable $X$ with parameter
	$\mu$ is defined by the following probability distribution on $j=0,1,...,n$:
	\[
	\Pr(X=j)=\frac{e^{-\mu}\mu^{j}}{j!}.
	\]
	When throwing $m$ balls into $n$ bins, the distribution of the number
	of balls in a bin is approximately Poisson with $\mu=m/n$ which is
	exactly the average number of balls per bin.
	\item [{Bounds~on~Poisson~Random~Variables.}] Let $X$ be a Poisson
	random variable with parameter $\mu$.
	
	\begin{enumerate}
		\item If $x>\mu$, then 
		\[
		\Pr(X\geq x)\leq\frac{e^{-\mu}(e\mu)^{x}}{x^{x}}.
		\]
		\item If $x<\mu$, then
		\[
		\Pr(X\leq x)\leq\frac{e^{-\mu}(e\mu)^{x}}{x^{x}}.
		\]
	\end{enumerate}
	\item [{Properties~of~Expectations.}] 
	
	\begin{itemize}
		\item If $X$ is a random variable and $a$ is a constant, then $\E[aX]=a\E[X]$.
		Thus, $\mathrm{\E}[\E[X]]=\E[X]$, because $\E[X]$ is a constant.
		\item Linearity of expectation: For any set of random variables $X_{1},...,X_{n}$,
		we have $\E[\sum_{i=1}^{n}X_{i}]=\sum_{i=1}^{n}\E[X_{i}]$.
	\end{itemize}
\end{description}

\section*{Concentration Bounds}
\begin{description}
	\item [{Union~Bound~(Boole's~Inequality).}] The probability that at
	least one event in a set of events happens is no greater than the
	sum of the probabilities of the individial events, i.e., for a set
	of events $A_{1},...,A_{n}$,
	\[
	\Pr\Big(\bigcup_{i=1}^{n}A_{i}\Big)\leq\sum_{i=1}^{n}\Pr(A_{i}).
	\]
	\item [{Markov's~Inequality.}] Let $X$ be a random variable that assumes
	only non-negative values. Then, for all $a>0$,
	\[
	\Pr(X\geq a)\leq\frac{\E[X]}{a}.
	\]
	\item [{Variance.}] The variance of a random variable $X$ is defined as\enspace{}$\Var[X]=\E[X^{2}]-(\E[X])^{2}.$
	
	If $X$ is a binomial random variable, then\enspace{}$\Var[X]=np(1-p).$
	
	If $X$ is a geometric random variable, then\enspace{}$\Var[X]=(1-p)/p^{2}.$
	\item [{Chebyshev's~Inequality.}] For any $a>0$,
	\[
	\Pr(|X-\E[X]|\geq a)\leq\frac{\mathrm{Var}[X]}{a^{2}}.
	\]
	\item [{Chernoff~Bounds.}] Let $X_{1},...,X_{n}$ be independent Poisson
	trials such that $\Pr(X_{i})=p_{i}$. Let $X=\sum_{i=1}^{n}X_{i}$,
	$\mu=\E[X]$, and $\mu_{L}\leq\mu\leq\mu_{H}$. The following Chernoff
	bounds hold:
	
	\begin{center}
		\begin{tabular}{lll}
			For $\delta>0$, & $\Pr(X\geq(1+\delta)\mu)\leq\left(\frac{e^{\delta}}{(1+\delta)^{(1+\delta)}}\right)^{\mu}$\bigskip{}
			& \cite[Page 64]{Michael2005}\tabularnewline
			For $0<\delta<1$, & $\Pr(X\leq(1-\delta)\mu)\leq\left(\frac{e^{-\delta}}{(1-\delta)^{(1-\delta)}}\right)^{\mu}$\bigskip{}
			& \cite[Page 66]{Michael2005}\tabularnewline
			For $0<\delta<1$, & $\Pr(X\geq(1+\delta)\mu)\leq e^{-\mu\delta^{2}/3}$\bigskip{}
			& \cite[Page 64]{Michael2005}\tabularnewline
			For $0<\delta<1$, & $\Pr(X\leq(1-\delta)\mu)\leq e^{-\mu\delta^{2}/2}$\bigskip{}
			& \cite[Page 66]{Michael2005}\tabularnewline
			For $0<\delta<1$, & $\Pr(|X-\mu|\geq\delta\mu)\leq2e^{-\mu\delta^{2}/3}$\bigskip{}
			& \cite[Page 67]{Michael2005}\tabularnewline
			For $\delta>0$, & $\Pr(X<\mu-\delta),\,\,\Pr(X>\mu+\delta)\leq e^{-2\delta^{2}/n}$\bigskip{}
			& \cite[Page 6]{dubhashi:2009}\tabularnewline
			For $\delta>0$, & $\Pr(X<\mu_{L}-\delta),\,\,\Pr(X>\mu_{H}+\delta)\leq e^{-2\delta^{2}/n}$\bigskip{}
			& \cite[Page 8]{dubhashi:2009}\tabularnewline
			For $\delta>2e\mu$, & $\Pr(X>\delta)\leq2^{-\delta}$\bigskip{}
			& \cite[Page 7]{dubhashi:2009}\tabularnewline
			For $\delta>1$, & $\Pr(X\geq(1+\delta)\mu)\leq e^{-\mu\delta/3}$ & Wikipedia\tabularnewline
		\end{tabular}
		\par\end{center}
	\item [{Convex~Function.}] A function $f:\mathbb{R\to\mathbb{R}}$ is
	said to be convex if and only if, for any $x_{1},x_{2}$ and $0\leq\lambda\leq1$,
	\[
	f(\lambda x_{1}+(1-\lambda)x_{2})\leq\lambda f(x_{1})+(1-\lambda)f(x_{2}).
	\]
	\item [{Jensen's~Inequality.}] If $f$ is a convex function, then\enspace{}$\E[f(X)]\geq f(\E[X]).$
	\item [{Conditional~Expectation.}] A conditional expectation is not a
	scalar value; it is a random variable defined as
	
	\[
	\E[X|Y]=\{\E[X|Y=y]\;\text{with}\;\Pr(Y=y)\}.
	\]
	In other words, $\E[X|Y]$ is a function that maps every $y\in Y$
	to a $x\in X$. It is easy to see that
	
	\[
	\E[\E[X|Y]]=\E[X].
	\]
	Also,
	
	\[
	\E[\E[X|Y,Z]|Z]=\E[X|Z].
	\]
	
\end{description}

\section*{Martingales}

A martingale is a sequence of random variables used to model an event
that is a function of a sequence of past events. Martingales are useful
for events where the knowledge of past events does not allow to predict
the expected value or the actual probability of the event. 

For example, suppose that we are throwing $m$ balls independently
and uniformly at random into $n$ bins. Let $X_{i}$ be the random
variable representing the bin into which the $i$-th ball falls, and
$F$ be the number of empty bins after the $m$ balls are thrown.
Then, the sequence 
\[
Z_{i}=\mathrm{E}[F|X_{0},...,X_{i}]
\]
is a martingale. This model allows us to find a concentration bound
for $F$ using Azuma's inequality (which we will describe later):
\[
\Pr(|F-\mathrm{E}[F]|\geq\epsilon)\leq2e^{-2\epsilon^{2}/m}.
\]

\begin{definition}
	A sequence of random variables $Z_{0},Z_{1},...$ is a \emph{martingale
		with respect to a sequence $X_{0},X_{1}...$,} if for all $i\geq0$,
	the following conditions hold:
	
	\begin{enumerate}
		\item $Z_{i}$ is a function of $X_{0},...,X_{i}$;
		\item $\E[|Z_{i}|]<\infty$; and
		\item $\E[Z_{i+1}|X_{0},...,X_{i}]=Z_{i}$.
	\end{enumerate}
\end{definition}
~
\begin{definition}
	A sequence of random variables $Z_{0},Z_{1},...$ is called a \emph{martingale}
	when it is a martingale with respect to itself, that is
	
	\begin{enumerate}
		\item $\E[|Z_{i}|]<\infty$ 
		\item $\E[Z_{i+1}|Z_{0},...,Z_{i}]=Z_{i}$
	\end{enumerate}
\end{definition}
~
\begin{definition}
	Let $X_{0},...,X_{n}$ be a sequence of random variables and $Y$
	be a random variable that depends on the $X_{i}$'s. Then, for $i=0,1,...n$,
	\[
	Z_{i}=\E[Y|X_{0},...,X_{i}]
	\]
	is a martingale with respect to $X_{0},...,X_{n}$ and is called
	a \emph{Doob martingale}.
\end{definition}
~
\begin{definition}
	A random variable $T$ is a stopping time for $\{Z_{i}\}$ if the
	event $T=n$ depends only on $Z_{0},...,Z_{n}$.
\end{definition}
\begin{theorem}
	\emph{{[}Martingale Stopping Time Theorem{]}} If $Z_{0},Z_{1},...$
	is a martingale with respect to $X_{0},X_{1},...$ and $T$ is a stopping
	time for $X_{1},X_{2},...$, then 
	\[
	\E[Z_{T}]=\E[Z_{0}],
	\]
	
	if one of the following holds:
	
	\begin{enumerate}
		\item $|Z_{i}|<c$ for some constant $c$.
		\item $T$ is bounded.
		\item $\E[T]<\infty$ and there exists a $c$ such that $E[Z_{i+1}-Z_{i}|X_{1},...,X_{i}]<c$.
	\end{enumerate}
\end{theorem}
~
\begin{theorem}
	\emph{{[}Azuma's Inequality{]}} Let $X_{0},...,X_{n}$ be a martingale
	such that $|X_{k}-X_{k-1}|\leq c_{k}$. Then, for all $t\geq0$ and
	any $\lambda>0$,
	\[
	\Pr(|X_{t}-X_{0}|\geq\lambda)\leq2e^{-\lambda^{2}/(2\sum_{k=1}^{t}c_{k}^{2})}.
	\]
\end{theorem}
~
\begin{theorem}
	\emph{{[}Azuma's Inequality \textendash{} Tighter Bound{]} }Let $X_{0},...,X_{n}$
	be a martingale such that 
	\[
	B_{k}\leq X_{k}-X_{k-1}\leq B_{k}+d_{k}
	\]
	for some constants $d_{k}$ and for some random variables $B_{k}$.
	Then, for all $t\geq0$ and any $\lambda>0$,
	\[
	\Pr(|X_{t}-X_{0}|\geq\lambda)\leq2e^{-2\lambda^{2}/(\sum_{k=1}^{t}d_{k}^{2})}.
	\]
\end{theorem}
\begin{definition}
	{[}Informal{]} A \emph{dependency graph} for a set of events $E_{1},...,E_{n}$
	is graph with a vertex for each event, where there is an edge between
	two vertices iff they have some dependency with each other. In other
	words, if the two vertices are mutually independent, then there is
	no edge between them.
\end{definition}
\begin{theorem}
	\emph{{[}Lovasz Local Lemma{]} }Let $E_{1},...,E_{n}$ be a set of
	events and 
	
	\begin{enumerate}
		\item for all $i$, $\Pr(E_{i})\leq p$;
		\item \textup{\emph{the degree of the dependency graph of $E_{1},...,E_{n}$
				is at most $d$;}}
		\item $ep(d+1)\leq1$.
	\end{enumerate}
	Then,
	\[
	\Pr\big(\bigcap_{i=1}^{n}\bar{E_{i}}\big)>0.
	\]
	
\end{theorem}

\section*{Useful bounds}

The following bounds follow from the Taylor series expansion of $e^{x}$:

\[
1+x\leq e^{x}
\]

\[
1-x\leq e^{-x}
\]

\[
\text{{For\,}}x\geq0:\,\,1+x\geq e^{x}(1-x^{2})
\]

\[
\text{{For\,}}x\geq0:\,\,1-x\geq e^{-x}-x^{2}/2
\]

\bibliographystyle{plain}
\bibliography{../secbib/secbib}			% pull from https://github.com/mahdiz/secbib

\end{document}